\documentclass[a4paper,UKenglish]{lipics-v2016}
%This is a template for producing LIPIcs articles. 
%See lipics-manual.pdf for further information.
%for A4 paper format use option "a4paper", for US-letter use option "letterpaper"
%for british hyphenation rules use option "UKenglish", for american hyphenation rules use option "USenglish"
% for section-numbered lemmas etc., use "numberwithinsect"
 
\usepackage{microtype}%if unwanted, comment out or use option "draft"

%\graphicspath{{./graphics/}}%helpful if your graphic files are in another directory

\bibliographystyle{plainurl}% the recommended bibstyle

% Author macros::begin %%%%%%%%%%%%%%%%%%%%%%%%%%%%%%%%%%%%%%%%%%%%%%%%
\title{The Three-HITs Theorem\footnote{This work was partially supported by COST Action CA15123.}}
\titlerunning{The Three-HITs Theorem} %optional, in case that the title is too long; the running title should fit into the top page column

%% Please provide for each author the \author and \affil macro, even when authors have the same affiliation, i.e. for each author there needs to be the  \author and \affil macros
\author[1]{Andrej Bauer}
\author[2]{Niels van der Weide}
\affil[1]{Department of Mathematics and Physics, University of Ljubljana, Ljubljana, Slovenia\\
  \texttt{Andrej.Bauer@andrej.com}}
\affil[2]{Department of Computer Science, Radboud University, Nijmegen, The Netherlands\\
  \texttt{nweide@cs.ru.nl}}
\authorrunning{A. Bauer and N. van der Weide} %mandatory. First: Use abbreviated first/middle names. Second (only in severe cases): Use first author plus 'et. al.'

%\Copyright{Andrej Bauer and Niels van der Weide}%mandatory, please use full first names. LIPIcs license is "CC-BY";  http://creativecommons.org/licenses/by/3.0/

\subjclass{F.4.1 Mathematical Logic}% mandatory: Please choose ACM 1998 classifications from http://www.acm.org/about/class/ccs98-html . E.g., cite as "F.1.1 Models of Computation". 
\keywords{homotopy type theory, higher inductive types, category theory}% mandatory: Please provide 1-5 keywords
% Author macros::end %%%%%%%%%%%%%%%%%%%%%%%%%%%%%%%%%%%%%%%%%%%%%%%%%

%Editor-only macros:: begin (do not touch as author)%%%%%%%%%%%%%%%%%%%%%%%%%%%%%%%%%%
\EventEditors{John Q. Open and Joan R. Acces}
\EventNoEds{2}
\EventLongTitle{42nd Conference on Very Important Topics (CVIT 2016)}
\EventShortTitle{CVIT 2016}
\EventAcronym{CVIT}
\EventYear{2016}
\EventDate{December 24--27, 2016}
\EventLocation{Little Whinging, United Kingdom}
\EventLogo{}
\SeriesVolume{42}
\ArticleNo{23}
% Editor-only macros::end %%%%%%%%%%%%%%%%%%%%%%%%%%%%%%%%%%%%%%%%%%%%%%%



\usepackage[all,cmtip]{xy}
\usepackage{xcolor}
\usepackage{amssymb}
\usepackage{amsthm}
\usepackage{thmtools}



\declaretheorem[style=definition,qed=$\blacksquare$]{construction}


% lstlisting coq style (inspired from a file of Assia Mahboubi)
%
\lstdefinelanguage{Coq}{ 
%
% Anything betweeen $ becomes LaTeX math mode
mathescape=true,
%
% Comments may or not include Latex commands
texcl=false, 
%
% Vernacular commands
morekeywords=[1]{Section, Module, End, Require, Import, Export,
  Variable, Variables, Parameter, Parameters, Axiom, Hypothesis,
  Hypotheses, Notation, Local, Tactic, Reserved, Scope, Open, Close,
  Bind, Delimit, Definition, Let, Ltac, Fixpoint, CoFixpoint, Add,
  Morphism, Relation, Implicit, Arguments, Unset, Contextual,
  Strict, Prenex, Implicits, Inductive, CoInductive, Record,
  Structure, Canonical, Coercion, Context, Class, Global, Instance,
  Program, Infix, Theorem, Lemma, Corollary, Proposition, Fact,
  Remark, Example, Proof, Goal, Save, Qed, Defined, Hint, Resolve,
  Rewrite, View, Search, Show, Print, Printing, All, Eval, Check,
  Projections, inside, outside, Def},
%
% Gallina
morekeywords=[2]{forall, exists, exists2, fun, fix, cofix, struct,
  match, with, end, as, in, return, let, if, is, then, else, for, of,
  nosimpl, when},
%
% Sorts
morekeywords=[3]{Type, Prop, Set, true, false, option},
%
% Various tactics, some are std Coq subsumed by ssr, for the manual purpose
morekeywords=[4]{pose, set, move, case, elim, apply, clear, hnf,
  intro, intros, generalize, rename, pattern, after, destruct,
  induction, using, refine, inversion, injection, rewrite, congr,
  unlock, compute, ring, field, fourier, replace, fold, unfold,
  change, cutrewrite, simpl, have, suff, wlog, suffices, without,
  loss, nat_norm, assert, cut, trivial, revert, bool_congr, nat_congr,
  symmetry, transitivity, auto, split, left, right, autorewrite},
%
% Terminators
morekeywords=[5]{by, done, exact, reflexivity, tauto, romega, omega,
  assumption, solve, contradiction, discriminate},
%
% Control
morekeywords=[6]{do, last, first, try, idtac, repeat},
%
% Comments delimiters, we do turn this off for the manual
morecomment=[s]{(*}{*)},
%
% Spaces are not displayed as a special character
showstringspaces=false,
%
% String delimiters
morestring=[b]",
morestring=[d]’,
%
% Size of tabulations
tabsize=3,
%
% Enables ASCII chars 128 to 255
extendedchars=false,
%
% Case sensitivity
sensitive=true,
%
% Automatic breaking of long lines
breaklines=false,
%
% Default style fors listings
basicstyle=\small,
%
% Position of captions is bottom
captionpos=b,
%
% flexible columns
columns=[l]flexible,
%
% Style for (listings') identifiers
identifierstyle={\ttfamily\color{black}},
% Style for declaration keywords
keywordstyle=[1]{\ttfamily\color{violet}},
% Style for gallina keywords
keywordstyle=[2]{\ttfamily\color{green}},
% Style for sorts keywords
keywordstyle=[3]{\ttfamily\color{blue}},
% Style for tactics keywords
keywordstyle=[4]{\ttfamily\color{blue}},
% Style for terminators keywords
keywordstyle=[5]{\ttfamily\color{red}},
%Style for iterators
%keywordstyle=[6]{\ttfamily\color{dkpink}},
% Style for strings
stringstyle=\ttfamily,
% Style for comments
commentstyle={\ttfamily\color{dkgreen}},
%
%moredelim=**[is][\ttfamily\color{red}]{/&}{&/},
literate=
    {\\forall}{{\color{dkgreen}{$\forall\;$}}}1
    {\\exists}{{$\exists\;$}}1
    {<-}{{$\leftarrow\;$}}1
    {=>}{{$\Rightarrow\;$}}1
    {==}{{\code{==}\;}}1
    {==>}{{\code{==>}\;}}1
%    {:>}{{\code{:>}\;}}1
    {->}{{$\rightarrow\;$}}1
    {<->}{{$\leftrightarrow\;$}}1
    {<==}{{$\leq\;$}}1
    {\#}{{$^\star$}}1 
    {\\o}{{$\circ\;$}}1 
    {\@}{{$\cdot$}}1 
    {\/\\}{{$\wedge\;$}}1
    {\\\/}{{$\vee\;$}}1
    {++}{{\code{++}}}1
    {~}{{\ }}1
    {\@\@}{{$@$}}1
    {\\mapsto}{{$\mapsto\;$}}1
    {\\hline}{{\rule{\linewidth}{0.5pt}}}1
%
}[keywords,comments,strings]

\lstnewenvironment{coq}{\lstset{language=Coq}}{}

% pour inliner dans le texte
\def\coqe{\lstinline[language=Coq, basicstyle=\small]}
% pour inliner dans les tableaux / displaymath...
\def\coqes{\lstinline[language=Coq, basicstyle=\scriptsize]}

%%% Local Variables: 
%%% mode: latex
%%% Local IspellDict: british
%%% TeX-master: "main.tex"
%%% End: 1
%
\def\BPmessage{Proof Tree (bussproofs) style macros. Version 1.1.}
% bussproofs.sty.  Version 1.1
%     (c) 1994,1995,1996,2004,2005,2006, 2011. 
%     Copyright retained by Samuel R. Buss.
%
% ==== Legal statement: ====
% This work may be distributed and/or modified under the
% conditions of the LaTeX Project Public License, either version 1.3
% of this license or (at your option) any later version.
% The latest version of this license is in
%  http://www.latex-project.org/lppl.txt.
% and version 1.3 or later is part of all distributions of LaTeX
% version 2005/12/1 or later.
%
% This work has the LPPL maintenance status 'maintained'.
%
% The Current Maintainer of the work is Sam Buss.
%
% This work consists of bussproofs.sty.
%  =====
% Informal summary of legal situation:
%     This software may be used and distributed freely, except that
%     if you make changes, you must change the file name to be different
%     than bussproofs.sty to avoid compatibility problems.
% The terms of the LaTeX Public License are the legally controlling terms
%     and override any contradictory terms of the "informal situation".
%
%     Please report comments and bugs to sbuss@ucsd.edu.
%
%  Thanks to Felix Joachimski for making changes to let these macros
%   work in plain TeX in addition to LaTeX.  Nothing has been done
%   to see if they work in AMSTeX.  The comments below mostly
%   are written for LaTeX, however.
%  July 2004, version 0.7
%       - bug fix, right labels with descenders inserted too much space.
%         Thanks to Peter Smith for finding this bug,
%     see http://www.phil.cam.ac.uk/teaching_staff/Smith/LaTeX/
%  March 2005, version 0.8.
%   Added a default definition for \fCenter at Denis Kosygin's
%   suggestion.
%  September 2005, version 0.9.
%       Fixed some subtle spacing problems, by adding %'s to the end of
%   few lines where they were inadvertantly omitted.  Thanks to
%   Arnold Beckmann for finding and fixing this problem.
%  April 2006, version 0.9.1.  Updated comments and testbp2.tex file.
%       No change to the actual macros.
%  June 2006, version 1.0.  The first integer numbered release.
%       New feature: root of proof may now be at the bottom instead of
%       at just the top.  Thanks to Alex Hertel for the suggestion to implement
%       this.
%  June 2011, version 1.1.  
%       New feature: 4-ary and 5-ary inferences.  Thanks to Thomas Strathmann
%       for taking the initiative to implement these.
%         Four new commands: QuaternaryInf(C) and QuinaryInf(C).
%       Bug fix: \insertBetweenHyps now works for proofs with root at top and
%            three or more hypotheses..

% A good exposition of how to use bussproofs.sty (version 0.9) has been written
%  by Peter Smith and is available on the internet.
% The comments below also describe the features of bussproofs.sty,
%  including user-modifiable parameters.

%  bussproofs.sty allows the construction of proof trees in the
%     style of the sequent calculus and many other proof systems
%     One novel feature of these macros is they support the horizontal
%     alignment according to some center point specified with the
%     command \fCenter.  This is the style often used in sequent
%     calculus proofs.
%     Proofs are specified in left-to-right traversal order.
%       For example a proof
%               A   B
%                               -----
%                          D      C
%                         ---------
%                             E
%
%     if given in the order D,A,B,C,E.  Each line in the proof is
%     specified according to the arity of the inference which generates
%     it.  Thus, E would be specified with a \BinaryInf  or  \BinaryInfC
%     command.
%
%     The above proof tree could be displayed with the commands:
%
%               \AxiomC{D}
%               \AxiomC{A}
%               \AxiomC{B}
%               \BinaryInfC{C}
%               \BinaryInfC{E}
%               \DisplayProof
%
%     Inferences in a proof may be nullary (axioms), unary, binary, or
%     trinary.
%
%     IMPORTANT:  You must give the \DisplayProof command to make the proof
%      be printed.  To display a centered proof on a line by itself,
%      put the proof inside \begin{center} ... \end{center}.
%
%     There are two styles for specifying horizontal centering of
%     lines (formulas or sequents) in a proof.  One format \AxiomC{...}
%     just centers the formula {...} in the usual way.  The other
%     format is \Axiom$...\fCenter...$.  Here,  the \fCenter specifies
%     the center of the formula.  (It is permissable for \fCenter to
%     generate typeset material; in fact, I usually define it to generate
%     the sequent arrow.)  In unary inferences, the \fCenter
%     positions will be vertically aligned in the upper and lower lines of
%     the inference.  Unary, Binary, Trinary inferences are specified
%     with the same format as Axioms.  The two styles of centering
%     lines may be combined in a single proof.
%
%     By using the optional \EnableBpAbbreviations command, various
%   abbreviated two or three letter commands are enabled.  This allows,
%   in particular:
%   \AX and \AXC for \Axiom and \AxiomC, (resp.),
%   \DP for \DisplayProof,
%   \BI and \BIC for \BinaryInf and \BinaryInfC,
%   \UI and \UIC for \UnaryInf  and \UnaryInfC,
%   \TI and \TIC for \TrinaryInf and \TrinaryInfC,
%   \LL and \RL for \LeftLabel and \RightLabel.
%   See the source code below for additional abbreviations.
%     The enabling of these short abbreviations is OPTIONAL, since
%       there is the possibility of conflicting with names from other
%       macro packages.
%
%     By default, the inferences have single horizontal lines (scores)
%       This can be overridden using the \doubleLine, \noLine commands.
%       These two commands affect only the next inference.  You can make
%   make a permanent override that applies to the rest of the current
%       proof using \alwaysDoubleLine and \alwaysNoLine.  \singleLine
%   and \alwaysSingleLine work in the analogous way.
%
%     The macros do their best to give good placements of for the
%     parts of the proof.  Several macros allow you to override the
%     defaults.  These are \insertBetweenHyps{...} which overrides
%     the default spacing between hypotheses of Binary and Trinary
%     inferences with {...}.  And \kernHyps{...} specifies a distance
%     to shift the whole block of hypotheses to the right (modifying
%     the default center position.
%       Other macros set the vertical placement of the whole proof.
%     The default is to try to do a good job of placement for inferences
%     included in text.  Two other useful macros are: \bottomAlignProof
%     which aligns the hbox output by \DisplayProof according to the base
%     of the bottom line of the proof, and \centerAlignProof which
%     does a precise center vertical alignment.
%
%     Often, one wishes to place a label next to an inference, usually
%   to specify the type of inference.  These labels can be placed
%       by using the commands \LeftLabel{...} and \RightLabel{...}
%       immediately before the command which specifies the inference.
%       For example, to generate
%
%                       A     B
%                      --------- X
%                          C
%
%       use the commands
%             \AxiomC{A}
%             \AxiomC{B}
%             \RightLabel{X}
%             \BinaryInfC{C}
%             \DisplayProof
%
%     The \DisplayProof command just displays the proof as a text
%   item.  This allows you to put proofs anywhere normal text
%   might appear; for example, in a paragraph, in a table, in
%   a tabbing environment, etc.  When displaying a proof as inline text,
%   you should write \DisplayProof{}  (with curly brackets) so that
%   LaTeX will not "eat" the white space following the \DisplayProof
%   command.
%     For displaying proofs in a centered display:  Do not use the \[...\]
%   construction (nor $$...$$).  Instead use
%   \begin{center} ... \DisplayProof\end{center}.
%     Actually there is a better construction to use instead of the
%   \begin{center}...\DisplayProof\end{center}.  This is to
%   write
%       \begin{prooftree} ... \end{prooftree}.
%       Note there is no \DisplayProof used for this: the
%   \end{prooftree} automatically supplies the \DisplayProof
%   command.
%
%     Warning: Any commands that set line types or set vertical or
%   horizontal alignment that are given AFTER the \DisplayProof
%   command will affect the next proof, no matter how distant.




% Usages:
% =======
%
%   \Axiom$<antecedent>\fCenter<succedent>$
%
%   \AxiomC{<whole sequent or formula}
%
%   Note that the use of surrounding {}'s is mandatory in \AxiomC and
%   is prohibited in $\Axiom.  On the other hand, the $'s are optional
%   in \AxiomC and are mandatory in \Axiom.  To typeset the argument
%   to \AxiomC in math mode, you must use $'s (or \(...\) ).
%   The same comments apply to the inference commands below.
%
%   \UnaryInf$<antecendent>\fCenter<succedent>$
%
%   \UnaryInfC{<whole sequent or formula>}
%
%   \BinaryInf$<antecendent>\fCenter<succedent>$
%
%   \BinaryInfC{<whole sequent or formula>}
%
%   \TrinaryInf$<antecendent>\fCenter<succedent>$
%
%   \TrinaryInfC{<whole sequent or formula>}
%
%   \QuaternaryInf$<antecendent>\fCenter<succedent>$
%
%   \QuaternaryInfC{<whole sequent or formula>}
%
%   \QuinaryInf$<antecendent>\fCenter<succedent>$
%
%   \QuinaryInfC{<whole sequent or formula>}
%
%   \LeftLabel{<text>} - Puts <text> as a label to the left
%             of the next inference line.  (Works even if
%             \noLine is used too.)
%
%   \RightLabel{<text>} - Puts <text> as a label to the right of the
%             next inference line.  (Also works with \noLine.)
%
%   \DisplayProof - outputs the whole proof tree (and finishes it).
%           The proof tree is output as an hbox.
%
%
%   \kernHyps{<dimen>}  - Slides the upper hypotheses right distance <dimen>
%             (This is similar to shifting conclusion left)
%           - kernHyps works with Unary, Binary and Trinary
%             inferences and with centered or uncentered sequents.
%           - Negative values for <dimen> are permitted.
%
%   \insertBetweenHyps{...} - {...} will be inserted between the upper
%             hypotheses of a Binary or Trinary Inferences.
%                         It is possible to use negative horizontal space
%                 to push them closer together (and even overlap).
%             This command affects only the next inference.
%
%   \doubleLine         - Makes the current (ie, next) horizontal line doubled
%
%   \alwaysDoubleLine   - Makes lines doubled for rest of proof
%
%   \singleLine     - Makes the current (ie, next) line single
%
%   \alwaysSingleLine   - Undoes \alwaysDoubleLine or \alwaysNoLine.
%
%   \noLine     - Make no line at all at current (ie next) inference.
%
%   \alwaysNoLine       - Makes no lines for rest of proof. (Untested)
%
%   \solidLine      - Does solid horizontal line for current inference
%
%   \dottedLine     - Does dotted horizontal line for current inference
%
%   \dashedLine     - Does dashed horizontal line for current inference
%
%   \alwaysSolidLine    - Makes the indicated change in line type, permanently
%   \alwaysDashedLine       until end of proof or until overridden.
%   \alwaysDottedLine
%
%   \bottomAlignProof   - Vertically align proof according to its bottom line.
%   \centerAlignProof   - Vertically align proof proof precisely in its center.
%   \normalAlignProof   - Overrides earlier bottom/center AlignProof commands.
%             The default alignment will look good in most cases,
%               whether the proof is displayed or is
%               in-line.  Other alignments may be more
%               appropriate when putting proofs in tables or
%               pictures, etc.  For custom alignments, use
%               TeX's raise commands.
%
%   \rootAtTop  - specifies that proofs have their root a the top.  That it,
%                 proofs will be "upside down".
%   \rootAtBottom - (default) Specifies that proofs have root at the bottom
%         The \rootAtTop and \rootAtBottom commands apply *only* to the
%         current proof.  If you want to make them persistent, use one of
%         the next two commands:
%   \alwaysRootAtTop
%   \alwaysRootAtBottom (default)
%

% Optional short abbreviations for commands:
\def\EnableBpAbbreviations{%
    \let\AX\Axiom
    \let\AXC\AxiomC
    \let\UI\UnaryInf
    \let\UIC\UnaryInfC
    \let\BI\BinaryInf
    \let\BIC\BinaryInfC
    \let\TI\TrinaryInf
    \let\TIC\TrinaryInfC
    \let\QI\QuaternaryInf
    \let\QIC\QuaternaryInfC
    \let\QuI\QuinaryInf
    \let\QuIC\QuinaryInfC
    \let\LL\LeftLabel
    \let\RL\RightLabel
    \let\DP\DisplayProof
}

% Parameters which control the style of the proof trees.
% The user may wish to override these parameters locally or globally.
%   BUT DON'T CHANGE THE PARAMETERS BY CHANGING THIS FILE (to avoid
%   future incompatibilities).  Instead, you should change them in your
%   TeX document right after including this style file in the
%   header material of your LaTeX document.

\def\ScoreOverhang{4pt}         % How much underlines extend out
\def\ScoreOverhangLeft{\ScoreOverhang}
\def\ScoreOverhangRight{\ScoreOverhang}

\def\extraVskip{2pt}            % Extra space above and below lines
\def\ruleScoreFiller{\hrule}        % Horizontal rule filler.
\def\dottedScoreFiller{\hbox to4pt{\hss.\hss}}
\def\dashedScoreFiller{\hbox to2.8mm{\hss\vrule width1.4mm height0.4pt depth0.0pt\hss}}
\def\defaultScoreFiller{\ruleScoreFiller}  % Default horizontal filler.
\def\defaultBuildScore{\buildSingleScore}  % In \singleLine mode at start.

\def\defaultHypSeparation{\hskip.2in}   % Used if \insertBetweenHyps isn't given

\def\labelSpacing{3pt}      % Horizontal space separating labels and lines

\def\proofSkipAmount{\vskip.8ex plus.8ex minus.4ex}
            % Space above and below a prooftree display.

\def\defaultRootPosition{\buildRootBottom} % Default: Proofs root at bottom
%\def\defaultRootPosition{\buildRootTop}  % Makes all proofs upside down

\ifx\fCenter\undefined
\def\fCenter{\relax}
\fi

%
% End of user-modifiable parameters.
% %%%%%%%%%%%%%%%%%%%%%%%%%%%%%%%%%%%%%%%%%%%%%%%%%%%%%%%%%%%%%%%%%%%%%%

% Here are some internal paramenters and defaults.  Not really intended
%  to be user-modifiable.

\def\theHypSeparation{\defaultHypSeparation}
\def\alwaysScoreFiller{\defaultScoreFiller} % Horizontal filler.
\def\alwaysBuildScore{\defaultBuildScore}
\def\theScoreFiller{\alwaysScoreFiller} % Horizontal filler.
\def\buildScore{\alwaysBuildScore}   %This command builds the score.
\def\hypKernAmt{0pt}    % Initial setting for kerning the hypotheses.

\def\defaultLeftLabel{}
\def\defaultRightLabel{}

\def\myTrue{Y}
\def\bottomAlignFlag{N}
\def\centerAlignFlag{N}
\def\defaultRootAtBottomFlag{Y}
\def\rootAtBottomFlag{Y}

% End of internal parameters and defaults.

\expandafter\ifx\csname newenvironment\endcsname\relax%
% If in TeX:
\message{\BPmessage}
\def\makeatletter{\catcode`\@=11\relax}
\def\makeatother{\catcode`\@=12\relax}
\makeatletter
\def\newcount{\alloc@0\count\countdef\insc@unt}
\def\newdimen{\alloc@1\dimen\dimendef\insc@unt}
\def\newskip{\alloc@2\skip\skipdef\insc@unt}
\def\newbox{\alloc@4\box\chardef\insc@unt}
\makeatother
\else
% If in LaTeX
\typeout{\BPmessage}
\newenvironment{prooftree}%
{\begin{center}\proofSkipAmount \leavevmode}%
{\DisplayProof \proofSkipAmount \end{center} }
\fi

\def\thecur#1{\csname#1\number\theLevel\endcsname}

\newcount\theLevel    % This counter is the height of the stack.
\global\theLevel=0      % Initialized to zero
\newcount\myMaxLevel
\global\myMaxLevel=0
\newbox\myBoxA      % Temporary storage boxes
\newbox\myBoxB
\newbox\myBoxC
\newbox\myBoxD
\newbox\myBoxLL     % Boxes for the left label and the right label.
\newbox\myBoxRL
\newdimen\thisAboveSkip     %Internal use: amount to skip above line
\newdimen\thisBelowSkip     %Internal use: amount to skip below line
\newdimen\newScoreStart     % More temporary storage.
\newdimen\newScoreEnd
\newdimen\newCenter
\newdimen\displace
\newdimen\leftLowerAmt%     Amount to lower left label
\newdimen\rightLowerAmt%    Amount to lower right label
\newdimen\scoreHeight%      Score height
\newdimen\scoreDepth%       Score Depth
\newdimen\htLbox%
\newdimen\htRbox%
\newdimen\htRRbox%
\newdimen\htRRRbox%
\newdimen\htAbox%
\newdimen\htCbox%

\setbox\myBoxLL=\hbox{\defaultLeftLabel}%
\setbox\myBoxRL=\hbox{\defaultRightLabel}%

\def\allocatemore{%
    \ifnum\theLevel>\myMaxLevel%
        \expandafter\newbox\curBox%
        \expandafter\newdimen\curScoreStart%
        \expandafter\newdimen\curCenter%
        \expandafter\newdimen\curScoreEnd%
        \global\advance\myMaxLevel by1%
    \fi%
}

\def\prepAxiom{%
    \advance\theLevel by1%
    \edef\curBox{\thecur{myBox}}%
    \edef\curScoreStart{\thecur{myScoreStart}}%
    \edef\curCenter{\thecur{myCenter}}%
    \edef\curScoreEnd{\thecur{myScoreEnd}}%
    \allocatemore%
}

\def\Axiom$#1\fCenter#2${%
    % Get level and correct names set.
    \prepAxiom%
    % Define the boxes
    \setbox\myBoxA=\hbox{$\mathord{#1}\fCenter\mathord{\relax}$}%
    \setbox\myBoxB=\hbox{$#2$}%
    \global\setbox\curBox=%
         \hbox{\hskip\ScoreOverhangLeft\relax%
        \unhcopy\myBoxA\unhcopy\myBoxB\hskip\ScoreOverhangRight\relax}%
    % Set the relevant dimensions for the boxes
    \global\curScoreStart=0pt \relax
    \global\curScoreEnd=\wd\curBox \relax
    \global\curCenter=\wd\myBoxA \relax
    \global\advance \curCenter by \ScoreOverhangLeft%
    \ignorespaces
}

\def\AxiomC#1{      % Note argument not in math mode
    % Get level and correct names set.
    \prepAxiom%
        % Define the box.
    \setbox\myBoxA=\hbox{#1}%
    \global\setbox\curBox =%
        \hbox{\hskip\ScoreOverhangLeft\relax%
                        \unhcopy\myBoxA\hskip\ScoreOverhangRight\relax}%
    % Set the relevant dimensions for the boxes
        \global\curScoreStart=0pt \relax
        \global\curScoreEnd=\wd\curBox \relax
        \global\curCenter=.5\wd\curBox \relax
        \global\advance \curCenter by \ScoreOverhangLeft%
    \ignorespaces
}

\def\prepUnary{%
    \ifnum \theLevel<1
        \errmessage{Hypotheses missing!}
    \fi%
    \edef\curBox{\thecur{myBox}}%
    \edef\curScoreStart{\thecur{myScoreStart}}%
    \edef\curCenter{\thecur{myCenter}}%
    \edef\curScoreEnd{\thecur{myScoreEnd}}%
}

\def\UnaryInf$#1\fCenter#2${%
    \prepUnary%
    \buildConclusion{#1}{#2}%
    \joinUnary%
    \resetInferenceDefaults%
    \ignorespaces%
}

\def\UnaryInfC#1{
    \prepUnary%
    \buildConclusionC{#1}%
    %Align and join the curBox and the new box into one vbox.
    \joinUnary%
    \resetInferenceDefaults%
    \ignorespaces%
}

\def\prepBinary{%
    \ifnum\theLevel<2
        \errmessage{Hypotheses missing!}
    \fi%
    \edef\rcurBox{\thecur{myBox}}%   Set up names of right hypothesis
    \edef\rcurScoreStart{\thecur{myScoreStart}}%
    \edef\rcurCenter{\thecur{myCenter}}%
    \edef\rcurScoreEnd{\thecur{myScoreEnd}}%
    \advance\theLevel by-1%
    \edef\lcurBox{\thecur{myBox}}% Set up names of left hypothesis
    \edef\lcurScoreStart{\thecur{myScoreStart}}%
    \edef\lcurCenter{\thecur{myCenter}}%
    \edef\lcurScoreEnd{\thecur{myScoreEnd}}%
}

\def\BinaryInf$#1\fCenter#2${%
    \prepBinary%
    \buildConclusion{#1}{#2}%
    \joinBinary%
    \resetInferenceDefaults%
    \ignorespaces%
}

\def\BinaryInfC#1{%
    \prepBinary%
    \buildConclusionC{#1}%
    \joinBinary%
    \resetInferenceDefaults%
    \ignorespaces%
}

\def\prepTrinary{%
    \ifnum\theLevel<3
        \errmessage{Hypotheses missing!}
    \fi%
    \edef\rcurBox{\thecur{myBox}}%   Set up names of right hypothesis
    \edef\rcurScoreStart{\thecur{myScoreStart}}%
    \edef\rcurCenter{\thecur{myCenter}}%
    \edef\rcurScoreEnd{\thecur{myScoreEnd}}%
    \advance\theLevel by-1%
    \edef\ccurBox{\thecur{myBox}}% Set up names of center hypothesis
    \edef\ccurScoreStart{\thecur{myScoreStart}}%
    \edef\ccurCenter{\thecur{myCenter}}%
    \edef\ccurScoreEnd{\thecur{myScoreEnd}}%
    \advance\theLevel by-1%
    \edef\lcurBox{\thecur{myBox}}% Set up names of left hypothesis
    \edef\lcurScoreStart{\thecur{myScoreStart}}%
    \edef\lcurCenter{\thecur{myCenter}}%
    \edef\lcurScoreEnd{\thecur{myScoreEnd}}%
}

\def\TrinaryInf$#1\fCenter#2${%
    \prepTrinary%
    \buildConclusion{#1}{#2}%
    \joinTrinary%
    \resetInferenceDefaults%
    \ignorespaces%
}

\def\TrinaryInfC#1{%
    \prepTrinary%
    \buildConclusionC{#1}%
    \joinTrinary%
    \resetInferenceDefaults%
    \ignorespaces%
}

%%%%%%%%%%%%%%%%%%%%%%%%%%%%%%%%%%%%%%%%%%%%%%%%%%%%%%%%
\def\prepQuaternary{%
    \ifnum\theLevel<4
        \errmessage{Hypotheses missing!}
    \fi%
    \edef\rrcurBox{\thecur{myBox}}%   Set up names of very right hypothesis
    \edef\rrcurScoreStart{\thecur{myScoreStart}}%
    \edef\rrcurCenter{\thecur{myCenter}}%
    \edef\rrcurScoreEnd{\thecur{myScoreEnd}}%
    \advance\theLevel by-1%
    \edef\rcurBox{\thecur{myBox}}%   Set up names of right hypothesis
    \edef\rcurScoreStart{\thecur{myScoreStart}}%
    \edef\rcurCenter{\thecur{myCenter}}%
    \edef\rcurScoreEnd{\thecur{myScoreEnd}}%
    \advance\theLevel by-1%
    \edef\ccurBox{\thecur{myBox}}% Set up names of center hypothesis
    \edef\ccurScoreStart{\thecur{myScoreStart}}%
    \edef\ccurCenter{\thecur{myCenter}}%
    \edef\ccurScoreEnd{\thecur{myScoreEnd}}%
    \advance\theLevel by-1%
    \edef\lcurBox{\thecur{myBox}}% Set up names of left hypothesis
    \edef\lcurScoreStart{\thecur{myScoreStart}}%
    \edef\lcurCenter{\thecur{myCenter}}%
    \edef\lcurScoreEnd{\thecur{myScoreEnd}}%
}

\def\QuaternaryInf$#1\fCenter#2${%
    \prepQuaternary%
    \buildConclusion{#1}{#2}%
    \joinQuaternary%
    \resetInferenceDefaults%
    \ignorespaces%
}

\def\QuaternaryInfC#1{%
    \prepQuaternary%
    \buildConclusionC{#1}%
    \joinQuaternary%
    \resetInferenceDefaults%
    \ignorespaces%
}

\def\joinQuaternary{% Construct the quarterary inference into a vbox.
    % Join the four hypotheses's boxes into one hbox.
    \setbox\myBoxA=\hbox{\theHypSeparation}%
    \lcurScoreEnd=\rrcurScoreEnd%
    \advance\lcurScoreEnd by\wd\rcurBox%
    \advance\lcurScoreEnd by\wd\lcurBox%
    \advance\lcurScoreEnd by\wd\ccurBox%
    \advance\lcurScoreEnd by3\wd\myBoxA%
    \displace=\lcurScoreEnd%
    \advance\displace by -\lcurScoreStart%
    \lcurCenter=.5\displace%
    \advance\lcurCenter by\lcurScoreStart%
    \ifx\rootAtBottomFlag\myTrue%
        \setbox\lcurBox=%
            \hbox{\box\lcurBox\unhcopy\myBoxA\box\ccurBox%
                      \unhcopy\myBoxA\box\rcurBox
                      \unhcopy\myBoxA\box\rrcurBox}%
    \else%
        \htLbox = \ht\lcurBox%
        \htAbox = \ht\myBoxA%
        \htCbox = \ht\ccurBox%
        \htRbox = \ht\rcurBox%
        \htRRbox = \ht\rrcurBox%
        \setbox\lcurBox=%
            \hbox{\lower\htLbox\box\lcurBox%
                  \lower\htAbox\copy\myBoxA\lower\htCbox\box\ccurBox%
                  \lower\htAbox\copy\myBoxA\lower\htRbox\box\rcurBox%
                  \lower\htAbox\copy\myBoxA\lower\htRRbox\box\rrcurBox}%
    \fi%
    % Adjust center of upper hypotheses according to how much
    %   the lower sequent is off-center.
    \displace=\newCenter%
    \advance\displace by -.5\newScoreStart%
    \advance\displace by -.5\newScoreEnd%
    \advance\lcurCenter by \displace%
    %Align and join the curBox and the two hypotheses's box into one vbox.
    \edef\curBox{\lcurBox}%
    \edef\curScoreStart{\lcurScoreStart}%
    \edef\curScoreEnd{\lcurScoreEnd}%
    \edef\curCenter{\lcurCenter}%
    \joinUnary%
}

%%%%%%%%%%%%%%%%%%%%%%%%%%%%%%%%%%%%%%%%%%%%%%%%%%%%%%%%
\def\prepQuinary{%
    \ifnum\theLevel<5
        \errmessage{Hypotheses missing!}
    \fi%
    \edef\rrrcurBox{\thecur{myBox}}%   Set up names of very very right hypothesis
    \edef\rrrcurScoreStart{\thecur{myScoreStart}}%
    \edef\rrrcurCenter{\thecur{myCenter}}%
    \edef\rrrcurScoreEnd{\thecur{myScoreEnd}}%
    \advance\theLevel by-1%
    \edef\rrcurBox{\thecur{myBox}}%   Set up names of very right hypothesis
    \edef\rrcurScoreStart{\thecur{myScoreStart}}%
    \edef\rrcurCenter{\thecur{myCenter}}%
    \edef\rrcurScoreEnd{\thecur{myScoreEnd}}%
    \advance\theLevel by-1%
    \edef\rcurBox{\thecur{myBox}}%   Set up names of right hypothesis
    \edef\rcurScoreStart{\thecur{myScoreStart}}%
    \edef\rcurCenter{\thecur{myCenter}}%
    \edef\rcurScoreEnd{\thecur{myScoreEnd}}%
    \advance\theLevel by-1%
    \edef\ccurBox{\thecur{myBox}}% Set up names of center hypothesis
    \edef\ccurScoreStart{\thecur{myScoreStart}}%
    \edef\ccurCenter{\thecur{myCenter}}%
    \edef\ccurScoreEnd{\thecur{myScoreEnd}}%
    \advance\theLevel by-1%
    \edef\lcurBox{\thecur{myBox}}% Set up names of left hypothesis
    \edef\lcurScoreStart{\thecur{myScoreStart}}%
    \edef\lcurCenter{\thecur{myCenter}}%
    \edef\lcurScoreEnd{\thecur{myScoreEnd}}%
}

\def\QuinaryInf$#1\fCenter#2${%
    \prepQuinary%
    \buildConclusion{#1}{#2}%
    \joinQuinary%
    \resetInferenceDefaults%
    \ignorespaces%
}

\def\QuinaryInfC#1{%
    \prepQuinary%
    \buildConclusionC{#1}%
    \joinQuinary%
    \resetInferenceDefaults%
    \ignorespaces%
}

\def\joinQuinary{% Construct the quinary inference into a vbox.
    % Join the five hypotheses's boxes into one hbox.
    \setbox\myBoxA=\hbox{\theHypSeparation}%
    \lcurScoreEnd=\rrrcurScoreEnd%
    \advance\lcurScoreEnd by\wd\rrcurBox%
    \advance\lcurScoreEnd by\wd\rcurBox%
    \advance\lcurScoreEnd by\wd\lcurBox%
    \advance\lcurScoreEnd by\wd\ccurBox%
    \advance\lcurScoreEnd by4\wd\myBoxA%
    \displace=\lcurScoreEnd%
    \advance\displace by -\lcurScoreStart%
    \lcurCenter=.5\displace%
    \advance\lcurCenter by\lcurScoreStart%
    \ifx\rootAtBottomFlag\myTrue%
        \setbox\lcurBox=%
            \hbox{\box\lcurBox\unhcopy\myBoxA\box\ccurBox%
                      \unhcopy\myBoxA\box\rcurBox
                      \unhcopy\myBoxA\box\rrcurBox
                      \unhcopy\myBoxA\box\rrrcurBox}%
    \else%
        \htLbox = \ht\lcurBox%
        \htAbox = \ht\myBoxA%
        \htCbox = \ht\ccurBox%
        \htRbox = \ht\rcurBox%
        \htRRbox = \ht\rrcurBox%
        \htRRRbox = \ht\rrrcurBox%
        \setbox\lcurBox=%
            \hbox{\lower\htLbox\box\lcurBox%
                  \lower\htAbox\copy\myBoxA\lower\htCbox\box\ccurBox%
                  \lower\htAbox\copy\myBoxA\lower\htRbox\box\rcurBox%
                  \lower\htAbox\copy\myBoxA\lower\htRRbox\box\rrcurBox%
                  \lower\htAbox\copy\myBoxA\lower\htRRRbox\box\rrrcurBox}%
    \fi%
    % Adjust center of upper hypotheses according to how much
    %   the lower sequent is off-center.
    \displace=\newCenter%
    \advance\displace by -.5\newScoreStart%
    \advance\displace by -.5\newScoreEnd%
    \advance\lcurCenter by \displace%
    %Align and join the curBox and the two hypotheses's box into one vbox.
    \edef\curBox{\lcurBox}%
    \edef\curScoreStart{\lcurScoreStart}%
    \edef\curScoreEnd{\lcurScoreEnd}%
    \edef\curCenter{\lcurCenter}%
    \joinUnary%
}

%%%%%%%%%%%%%%%%%%%%%%%%%%%%%%%%%%%%%%%%%%%%%%%%%%%%%%%%

\def\buildConclusion#1#2{% Build lower sequent w/ center at \fCenter position.
    % Define the boxes
        \setbox\myBoxA=\hbox{$\mathord{#1}\fCenter\mathord{\relax}$}%
        \setbox\myBoxB=\hbox{$#2$}%
    % Put them together in \myBoxC
    \setbox\myBoxC =%
          \hbox{\hskip\ScoreOverhangLeft\relax%
        \unhcopy\myBoxA\unhcopy\myBoxB\hskip\ScoreOverhangRight\relax}%
    % Calculate the center of the \myBoxC string.
    \newScoreStart=0pt \relax%
    \newCenter=\wd\myBoxA \relax%
    \advance \newCenter by \ScoreOverhangLeft%
    \newScoreEnd=\wd\myBoxC%
}

\def\buildConclusionC#1{% Build lower sequent w/o \fCenter present.
        % Define the box.
    \setbox\myBoxA=\hbox{#1}%
    \setbox\myBoxC =%
        \hbox{\hbox{\hskip\ScoreOverhangLeft\relax%
                        \unhcopy\myBoxA\hskip\ScoreOverhangRight\relax}}%
    % Calculate kerning to line up centers
    \newScoreStart=0pt \relax%
    \newCenter=.5\wd\myBoxC \relax%
    \newScoreEnd=\wd\myBoxC%
        \advance \newCenter by \ScoreOverhangLeft%
}

\def\joinUnary{%Align and join \curBox and \myBoxC into a single vbox
    \global\advance\curCenter by -\hypKernAmt%
    \ifnum\curCenter<\newCenter%
        \displace=\newCenter%
        \advance \displace by -\curCenter%
        \kernUpperBox%
    \else%
        \displace=\curCenter%
        \advance \displace by -\newCenter%
        \kernLowerBox%
    \fi%
        \ifnum \newScoreStart < \curScoreStart %
        \global \curScoreStart = \newScoreStart \fi%
    \ifnum \curScoreEnd < \newScoreEnd %
        \global \curScoreEnd = \newScoreEnd \fi%
    % Leave room for the left label.
    \ifnum \curScoreStart<\wd\myBoxLL%
        \global\displace = \wd\myBoxLL%
        \global\advance\displace by -\curScoreStart%
        \kernUpperBox%
        \kernLowerBox%
    \fi%
    % Draw the score
    \buildScore%
    % Form the score and labels into a box.
    \buildScoreLabels%
    % Form the new box and its dimensions
    \ifx\rootAtBottomFlag\myTrue%
        \buildRootBottom%
    \else%
        \buildRootTop%
    \fi%
    \global \curScoreStart=\newScoreStart%
    \global \curScoreEnd=\newScoreEnd%
    \global \curCenter=\newCenter%
}

\def\buildRootBottom{%
    \global \setbox \curBox =%
        \vbox{\box\curBox%
            \vskip\thisAboveSkip \relax%
            \nointerlineskip\box\myBoxD%
            \vskip\thisBelowSkip \relax%
            \nointerlineskip\box\myBoxC}%
}

\def\buildRootTop{%
    \global \setbox \curBox =%
        \vbox{\box\myBoxC%
            \vskip\thisAboveSkip \relax%
            \nointerlineskip\box\myBoxD%
            \vskip\thisBelowSkip \relax%
            \nointerlineskip\box\curBox}%
}

\def\kernUpperBox{%
        \global\setbox\curBox =%
            \hbox{\hskip\displace\box\curBox}%
        \global\advance \curScoreStart by \displace%
        \global\advance \curScoreEnd by \displace%
        \global\advance\curCenter by \displace%
}

\def\kernLowerBox{%
        \global\setbox\myBoxC =%
            \hbox{\hskip\displace\unhbox\myBoxC}%
        \global\advance \newScoreStart by \displace%
        \global\advance \newScoreEnd by \displace%
        \global\advance\newCenter by \displace%
}

\def\joinBinary{% Construct the binary inference into a vbox.
    % Join the two hypotheses's boxes into one hbox.
    \setbox\myBoxA=\hbox{\theHypSeparation}%
    \lcurScoreEnd=\rcurScoreEnd%
    \advance\lcurScoreEnd by\wd\lcurBox%
    \advance\lcurScoreEnd by\wd\myBoxA%
    \displace=\lcurScoreEnd%
    \advance\displace by -\lcurScoreStart%
    \lcurCenter=.5\displace%
    \advance\lcurCenter by\lcurScoreStart%
    \ifx\rootAtBottomFlag\myTrue%
        \setbox\lcurBox=%
            \hbox{\box\lcurBox\unhcopy\myBoxA\box\rcurBox}%
    \else%
        \htLbox = \ht\lcurBox%
        \htAbox = \ht\myBoxA%
        \htRbox = \ht\rcurBox%
        \setbox\lcurBox=%
            \hbox{\lower\htLbox\box\lcurBox%
                  \lower\htAbox\box\myBoxA\lower\htRbox\box\rcurBox}%
    \fi%
    % Adjust center of upper hypotheses according to how much
    %   the lower sequent is off-center.
    \displace=\newCenter%
    \advance\displace by -.5\newScoreStart%
    \advance\displace by -.5\newScoreEnd%
    \advance\lcurCenter by \displace%
    %Align and join the curBox and the two hypotheses's box into one vbox.
    \edef\curBox{\lcurBox}%
    \edef\curScoreStart{\lcurScoreStart}%
    \edef\curScoreEnd{\lcurScoreEnd}%
    \edef\curCenter{\lcurCenter}%
    \joinUnary%
}

\def\joinTrinary{% Construct the trinary inference into a vbox.
    % Join the three hypotheses's boxes into one hbox.
    \setbox\myBoxA=\hbox{\theHypSeparation}%
    \lcurScoreEnd=\rcurScoreEnd%
    \advance\lcurScoreEnd by\wd\lcurBox%
    \advance\lcurScoreEnd by\wd\ccurBox%
    \advance\lcurScoreEnd by2\wd\myBoxA%
    \displace=\lcurScoreEnd%
    \advance\displace by -\lcurScoreStart%
    \lcurCenter=.5\displace%
    \advance\lcurCenter by\lcurScoreStart%
    \ifx\rootAtBottomFlag\myTrue%
        \setbox\lcurBox=%
            \hbox{\box\lcurBox\unhcopy\myBoxA\box\ccurBox%
                      \unhcopy\myBoxA\box\rcurBox}%
    \else%
        \htLbox = \ht\lcurBox%
        \htAbox = \ht\myBoxA%
        \htCbox = \ht\ccurBox%
        \htRbox = \ht\rcurBox%
        \setbox\lcurBox=%
            \hbox{\lower\htLbox\box\lcurBox%
                  \lower\htAbox\copy\myBoxA\lower\htCbox\box\ccurBox%
                  \lower\htAbox\copy\myBoxA\lower\htRbox\box\rcurBox}%
    \fi%
    % Adjust center of upper hypotheses according to how much
    %   the lower sequent is off-center.
    \displace=\newCenter%
    \advance\displace by -.5\newScoreStart%
    \advance\displace by -.5\newScoreEnd%
    \advance\lcurCenter by \displace%
    %Align and join the curBox and the two hypotheses's box into one vbox.
    \edef\curBox{\lcurBox}%
    \edef\curScoreStart{\lcurScoreStart}%
    \edef\curScoreEnd{\lcurScoreEnd}%
    \edef\curCenter{\lcurCenter}%
    \joinUnary%
}

\def\DisplayProof{%
    % Display (and purge) the proof tree.
    % Choose the appropriate vertical alignment.
    \ifnum \theLevel=1 \relax \else%x
        \errmessage{Proof tree badly specified.}%
    \fi%
    \edef\curBox{\thecur{myBox}}%
    \ifx\bottomAlignFlag\myTrue%
        \displace=0pt%
    \else%
        \displace=.5\ht\curBox%
        \ifx\centerAlignFlag\myTrue\relax
        \else%
                \advance\displace by -3pt%
        \fi%
    \fi%
    \leavevmode%
    \lower\displace\hbox{\copy\curBox}%
    \global\theLevel=0%
    \global\def\alwaysBuildScore{\defaultBuildScore}% Restore "always"
    \global\def\alwaysScoreFiller{\defaultScoreFiller}% Restore "always"
    \global\def\bottomAlignFlag{N}%
    \global\def\centerAlignFlag{N}%
    \resetRootPosition
    \resetInferenceDefaults%
    \ignorespaces
}

\def\buildSingleScore{% Make an hbox with a single score.
    \displace=\curScoreEnd%
    \advance \displace by -\curScoreStart%
    \global\setbox \myBoxD =%
        \hbox to \displace{\expandafter\xleaders\theScoreFiller\hfill}%
    %\global\setbox \myBoxD =%
        %\hbox{\hskip\curScoreStart\relax \box\myBoxD}%
}

\def\buildDoubleScore{% Make an hbox with a double score.
    \buildSingleScore%
    \global\setbox\myBoxD=%
        \hbox{\hbox to0pt{\copy\myBoxD\hss}\raise2pt\copy\myBoxD}%
}

\def\buildNoScore{% Make an hbox with no score (raise a little anyway)
    \global\setbox\myBoxD=\hbox{\vbox{\vskip1pt}}%
}

\def\doubleLine{%
    \gdef\buildScore{\buildDoubleScore}% Set next score to this type
    \ignorespaces
}
\def\alwaysDoubleLine{%
    \gdef\alwaysBuildScore{\buildDoubleScore}% Do double for rest of proof.
    \gdef\buildScore{\buildDoubleScore}% Set next score to be double
    \ignorespaces
}
\def\singleLine{%
    \gdef\buildScore{\buildSingleScore}% Set next score to be single
    \ignorespaces
}
\def\alwaysSingleLine{%
    \gdef\alwaysBuildScore{\buildSingleScore}% Do single for rest of proof.
    \gdef\buildScore{\buildSingleScore}% Set next score to be single
    \ignorespaces
}
\def\noLine{%
    \gdef\buildScore{\buildNoScore}% Set next score to this type
    \ignorespaces
}
\def\alwaysNoLine{%
    \gdef\alwaysBuildScore{\buildNoScore}%Do nolines for rest of proof.
    \gdef\buildScore{\buildNoScore}% Set next score to be blank
    \ignorespaces
}
\def\solidLine{%
    \gdef\theScoreFiller{\ruleScoreFiller}% Use solid horizontal line.
    \ignorespaces
}
\def\alwaysSolidLine{%
    \gdef\alwaysScoreFiller{\ruleScoreFiller}% Do solid for rest of proof
    \gdef\theScoreFiller{\ruleScoreFiller}% Use solid horizontal line.
    \ignorespaces
}
\def\dottedLine{%
    \gdef\theScoreFiller{\dottedScoreFiller}% Use dotted horizontal line.
    \ignorespaces
}
\def\alwaysDottedLine{%
    \gdef\alwaysScoreFiller{\dottedScoreFiller}% Do dotted for rest of proof
    \gdef\theScoreFiller{\dottedScoreFiller}% Use dotted horizontal line.
    \ignorespaces
}
\def\dashedLine{%
    \gdef\theScoreFiller{\dashedScoreFiller}% Use dashed horizontal line.
    \ignorespaces
}
\def\alwaysDashedLine{%
    \gdef\alwaysScoreFiller{\dashedScoreFiller}% Do dashed for rest of proof
    \gdef\theScoreFiller{\dashedScoreFiller}% Use dashed horizontal line.
    \ignorespaces
}
\def\kernHyps#1{%
    \gdef\hypKernAmt{#1}%
    \ignorespaces
}
\def\insertBetweenHyps#1{%
    \gdef\theHypSeparation{#1}%
    \ignorespaces
}

\def\centerAlignProof{%
    \def\centerAlignFlag{Y}%
    \def\bottomAlignFlag{N}%
    \ignorespaces
}
\def\bottomAlignProof{%
    \def\centerAlignFlag{N}%
    \def\bottomAlignFlag{Y}%
    \ignorespaces
}
\def\normalAlignProof{%
    \def\centerAlignFlag{N}%
    \def\bottomAlignFlag{N}%
    \ignorespaces
}

\def\LeftLabel#1{%
    \global\setbox\myBoxLL=\hbox{{#1}\hskip\labelSpacing}%
    \ignorespaces
}
\def\RightLabel#1{%
    \global\setbox\myBoxRL=\hbox{\hskip\labelSpacing #1}%
    \ignorespaces
}

\def\buildScoreLabels{%
    \scoreHeight = \ht\myBoxD%
    \scoreDepth = \dp\myBoxD%
    \leftLowerAmt=\ht\myBoxLL%
    \advance \leftLowerAmt by -\dp\myBoxLL%
    \advance \leftLowerAmt by -\scoreHeight%
    \advance \leftLowerAmt by \scoreDepth%
    \leftLowerAmt=.5\leftLowerAmt%
    \rightLowerAmt=\ht\myBoxRL%
    \advance \rightLowerAmt by -\dp\myBoxRL%
    \advance \rightLowerAmt by -\scoreHeight%
    \advance \rightLowerAmt by \scoreDepth%
    \rightLowerAmt=.5\rightLowerAmt%
    \displace = \curScoreStart%
    \advance\displace by -\wd\myBoxLL%
    \global\setbox\myBoxD =%
        \hbox{\hskip\displace%
            \lower\leftLowerAmt\copy\myBoxLL%
            \box\myBoxD%
            \lower\rightLowerAmt\copy\myBoxRL}%
    \global\thisAboveSkip = \ht\myBoxLL%
    \global\advance \thisAboveSkip by -\leftLowerAmt%
    \global\advance \thisAboveSkip by -\scoreHeight%
    \ifnum \thisAboveSkip<0 %
        \global\thisAboveSkip=0pt%
    \fi%
    \displace = \ht\myBoxRL%
    \advance \displace by -\rightLowerAmt%
    \advance \displace by -\scoreHeight%
    \ifnum \displace<0 %
        \displace=0pt%
    \fi%
    \ifnum \displace>\thisAboveSkip %
        \global\thisAboveSkip=\displace%
    \fi%
    \global\thisBelowSkip = \dp\myBoxLL%
    \global\advance\thisBelowSkip by \leftLowerAmt%
    \global\advance\thisBelowSkip by -\scoreDepth%
    \ifnum\thisBelowSkip<0 %
        \global\thisBelowSkip = 0pt%
    \fi%
    \displace = \dp\myBoxRL%
    \advance\displace by \rightLowerAmt%
    \advance\displace by -\scoreDepth%
    \ifnum\displace<0 %
        \displace = 0pt%
    \fi%
    \ifnum\displace>\thisBelowSkip%
        \global\thisBelowSkip = \displace%
    \fi%
    \global\thisAboveSkip = -\thisAboveSkip%
    \global\thisBelowSkip = -\thisBelowSkip%
    \global\advance\thisAboveSkip by\extraVskip% Extra space above line
    \global\advance\thisBelowSkip by\extraVskip% Extra space below line
}

\def\resetInferenceDefaults{%
    \global\def\theHypSeparation{\defaultHypSeparation}%
    \global\setbox\myBoxLL=\hbox{\defaultLeftLabel}%
    \global\setbox\myBoxRL=\hbox{\defaultRightLabel}%
    \global\def\buildScore{\alwaysBuildScore}%
    \global\def\theScoreFiller{\alwaysScoreFiller}%
    \gdef\hypKernAmt{0pt}% Restore to zero kerning.
}


\def\rootAtBottom{%
    \global\def\rootAtBottomFlag{Y}%
}

\def\rootAtTop{%
    \global\def\rootAtBottomFlag{N}%
}

\def\resetRootPosition{%
    \global\edef\rootAtBottomFlag{\defaultRootAtBottomFlag}
}

\def\alwaysRootAtBottom{%
    \global\def\defaultRootAtBottomFlag{Y}
    \rootAtBottom
}

\def\alwaysRootAtTop{%
    \global\def\defaultRootAtBottomFlag{N}
    \rootAtTop
}



\newcommand{\comment}[1]{-- #1 --}
\newcommand{\remove}[1]{}
\newcommand{\inn}{{\rm in}}

\newcommand{\conc}[0]{\bullet}

\newcommand{\zero}[0]{\textbf{0}}
\newcommand{\one}[0]{\textbf{1}}
\newcommand{\two}[0]{\textbf{2}}

\newcommand{\frec}[0]{\!\operatorname{rec}}
\newcommand{\rec}[0]{\operatorname{rec}}
\newcommand{\nonrec}[0]{\operatorname{nonrec}}

\newcommand{\ap}[0]{\operatorname{ap}}
\newcommand{\apd}[0]{\operatorname{apd}}

\newcommand{\term}[0]{\textbf{term}}
\newcommand{\Con}[0]{\textbf{Con}}
\newcommand{\Approx}[0]{\textbf{Approx}}

\newcommand{\refl}[0]{\operatorname{\textbf{refl}}}
\newcommand{\hpushout}[0]{\operatorname{\textbf{hpushout}}}
\newcommand{\hocolim}[0]{\operatorname{\textbf{hocolim}}}
\newcommand{\seg}[0]{\operatorname{\textbf{seg}}}
\newcommand{\inl}[0]{\operatorname{\textbf{inl}}}
\newcommand{\inr}[0]{\operatorname{\textbf{inr}}}
\newcommand{\glue}[0]{\operatorname{\textbf{glue}}}
\newcommand{\inc}[0]{\operatorname{\textbf{inc}}}
\newcommand{\com}[0]{\operatorname{\textbf{com}}}
\newcommand{\Type}[0]{\operatorname{\textsc{Type}}}
\newcommand{\pt}[0]{*}

\newcommand{\dak}[1]{\widehat{#1}}
\newcommand{\hatt}{\dak{t}}
\newcommand{\hatr}{\dak{r}}


\begin{document}

\maketitle

\begin{abstract}
Lorem ipsum dolor sit amet, consectetur adipiscing elit. Praesent convallis orci arcu, eu mollis dolor. Aliquam eleifend suscipit lacinia. Maecenas quam mi, porta ut lacinia sed, convallis ac dui. Lorem ipsum dolor sit amet, consectetur adipiscing elit. Suspendisse potenti. 
 \end{abstract}

\section{Introduction}
Higher inductive types (HITs) extend normal inductive types by allowing constructors for both points and paths, rather than just for the points.
While inductive types are freely generated from a signature, a higher inductive type is freely generated from a signature with equations.
Numerous examples and definitions of such types have been given in the literature \cite{altenkirch2016quotient,awodey2012inductive,basoldhigher,sojakova2015higher,hottbook}, but a definition with a good metatheory remains to be given.
As a step towards that goal, we simplify the definition given in \cite{basoldhigher} bringing it closer to the intuition and the intended meaning.

Philosophically, one sees an inductive type as a type `which is built step by step'.
One starts with the nonrecursive constructors, and then at each step new terms are added by applying the recursive constructors to the previously built terms.
This is explained by a theorem which says that inductive types are initial algebras for a functor \cite{dybjer1994inductive,dybjer2003induction} and another one about how these algebras are obtained \cite{adamek1974free}.
For higher inductive types one would expect a similar result which explains philosophically what they are.
The only difference between higher inductive types and normal inductive types is that during the construction of higher inductive types several identifications are made which are described by the path constructors.

The goal of this paper is to show that the higher inductive types defined in \cite{basoldhigher} can be generated from the interval, homotopy pushouts and colimits, and that formally justifies this intuition on higher inductive types.
More concretely, we will prove the following theorem.
\begin{theorem}[Three-HITs Theorem]
\label{thm:three-hit}
In Martin-L\"of Type Theory extended with an interval object, homotopy pushouts and colimits, we can define for each higher inductive type as in \cite{basoldhigher} a tye with the same introduction, elimination and computation rules.
\end{theorem}
Also, this generalizes the results in \cite{doorn2016constructing,kraus2014general,kraus2016constructions} where the result is showed for truncations.
With this result the metatheory of higher inductive types could be simplified significantly, because rather than a general class, one only needs to check metatheoretical properties for three types.

In Section \ref{sec:prelims} we shall discuss some of the required notions for this paper.
More concretely, we recall the syntax of higher inductive types, and as examples we give the interval, the homotopy pushout, and colimit.
Next we define the approximating sequence of a higher inductive type in Section \ref{sec:approximator}, and in Section \ref{sec:rules} we show that the colimit of this sequence satisfies the rules of the given HIT.
For this we need some lemmata which are proved in Section \ref{sec:Lemmata}.

\section{Preliminaries}
\label{sec:prelims}
Let us briefly recall the scheme of higher inductive types which we shall use \cite{basoldhigher}.
First of all, we need \emph{polynomial functors}.
\begin{definition}
Let $X$ be a variable.
Then a \emph{polynomial} is given by the following grammar.
\[
F, G ::= B : \Type \ | \ X \ | \ F \times G \ | \ F + G
\]
\end{definition}
A possible extension would be to use arbitrary containers as in \cite{abbott2005containers}, but we shall refrain to do so.
With that extension the given proof requires the axiom of choice which generally does not hold in type theory.
One can prove that polynomials are functors, and that given a type family $Y : T \rightarrow \Type$ and a polynomial $F$, we get a lift $\bar{F} : F \> T \rightarrow \Type$.
We will write $F \> f$ to denote the application of a polynomial $F$ to a map $f$.
Also, given $f : \prod (x : T), Y \> x$, one can define $\bar{F} \> f : \prod (x : H \> T), \bar{H} \> Y \> x$.
Precise definitions of these are given in \cite{basoldhigher}.

Next we define the notion of a constructor term.
\begin{definition}
Given are functions $c : A \> T \rightarrow T$.
Then we say $t$ is a \emph{constructor term} over $c$ if we can find polynomials $F$ and $G$ such that $x : F \> T \Vdash t : G \> T$ can be derived using the following rules.
\begin{gather*}
	\AxiomC{$t : B$}
	\AxiomC{$T$ does not occur in $B$}
	\BinaryInfC{$x:F \> T \Vdash t : B$}
	\bottomAlignProof
	\DisplayProof 
	 \quad
	\AxiomC{}
	\UnaryInfC{$x : F \> T \Vdash x : F \> T$}
	\bottomAlignProof
	\DisplayProof
	\quad
	\AxiomC{$x : F \> T \Vdash r : A \> T$}
	\UnaryInfC{$x : F \> T \Vdash c \> r : T$}
	\bottomAlignProof
	\DisplayProof
	\\[7pt]
	\AxiomC{$j \in \{1, 2\}$}
	\AxiomC{$x : F \> T \Vdash r : G_1 \> T \times G_2 \> T$}
	\BinaryInfC{$x : F \> T \Vdash\pi_j \> r : G_j \> T$}
	\bottomAlignProof
	\DisplayProof
	\quad
	\AxiomC{$j \in \{1, 2\}$}
	\AxiomC{$x : F \> T \Vdash r_j : G_j$}
	\BinaryInfC{$x : F \> T \Vdash (r_1, r_2) : G_1 \> T\times G_2 \> T$}
	\bottomAlignProof
	\DisplayProof
	\\[7pt]
	\AxiomC{$j \in \{1, 2\}$}
	\AxiomC{$x : F \> T \Vdash r : G_j \> T$}
	\BinaryInfC{$x : F \> T : \inn_j \> r : G_1 \> T + G_2 \> T$}
	\bottomAlignProof
	\DisplayProof
\end{gather*}
\end{definition}

Using constructor terms we give the following scheme of higher inductive types.
\begin{definition}
\label{def:hit}
A \emph{higher inductive type} is defined according to the following scheme
\lstset{language=Coq}
\begin{lstlisting}
Inductive $H$ :=
| $c$ : $A \> H \rightarrow H$
| $p_i$ : $\prod (x : B_i \> H), t_i = r_i$ $\quad (i = 1, \ldots, m)$
\end{lstlisting}
where $A$ and each $B_i$ are polynomials, and each $t_i$ and $r_i$ are constructor terms over $c$ of type $H$ with $x : B_i \> H$ as variable..
\end{definition}

Before we can give the rules for higher inductive types, we need to define the lift of a constructor term.
\begin{definition}
Given are a constructor $c : A \> H \rightarrow H$, a type family $Y : H \rightarrow \Type$, and a term
\[
 \vdash f : \prod (x: A \> H), \bar{A} \> Y \> x \rightarrow  Y(c \> x).
\]
For a constructor term $F \> H \Vdash r : G \> H$ we define the \emph{lift $\dak{r}$ of $r$} with type
\[
 x : F \> H, h_x : \bar{F} \> Y \> x \vdash \dak{r} : \bar{G} \> Y \> r
\]
by induction in $r$ as follows.
	\begin{align*}
		\dak{t} & := t
		& \dak{x} & := h_x &
		\dak{c_i \> r} & := f_i \> r \> \dak{r}\\
		\dak{\pi_j \> r} & := \pi_j \> \dak{r} &
		\dak{(r_1, r_2)} & := (\dak{r_1}, \dak{r_2})
		& \dak{\inn_j \> r} & := \dak{r}
	\end{align*}
\end{definition}

With all these notions we can give the introduction, elimination and computation rules of higher inductive types.
The introduction rules for $H$ as given in Definition \ref{def:hit} are
\[
c : A \> H \rightarrow H,
\]
\[
p_i : \prod (x : B_i \> H), t_i = r_i.
\]
We also have an elimination rule for which we use the lifting of constructor terms.
\begin{equation*}
	\AxiomC{$\vdash Y : H \rightarrow \Type$}
	\noLine
	\UnaryInfC{$\vdash f : \prod (x : A \> H), \bar{A} \> Y \> x \rightarrow Y \> (c \> x)$}
%	\noLine
	\AxiomC{$\vdash q_i : \prod (x : B_i \> H) (h_x : \bar{B}_i \> Y \> x), \hatt_i =_{p_j \, x} \hatr_j$}
	\BinaryInfC{$\vdash H\frec(f, q_1, \ldots, q_n) : \prod (x : H), Y \> x$}
	\DisplayProof
\end{equation*}
Let us abbreviate $H\frec(f, q_1, \ldots, q_n)$ by $H\frec$.
The type $H$ also has computation rules for each point $t : A \> H$
\begin{equation*}
H\frec \> (c_i \> t) \equiv f_i \> t \> (\bar{A} \> H\frec \> t),
\end{equation*}
and for each $a : B_i \> H$
\begin{equation*}
\apd \> H\frec \> (p_j \> a) \equiv q_j \> a \> (\bar{B}_i \> H\frec \> a).
\end{equation*}
Note that these equalities are definitional rather than propositional.

Let us finish by defining some examples of higher inductive types which will be used a lot in this paper.
The first one would be the interval
\lstset{language=Coq}
\begin{lstlisting}
Inductive $I^1$ :=
| $0$ : $I^1$
| $1$ : $I^1$
| $\seg$ : $0 = 1$
\end{lstlisting}
Note that for every type $A$ with inhabitants $x$ and $y$ we have a path $x = y$ iff we have a map $I^1$ to $A$ sending 0 and 1 to $x$ and $y$ respectively.
Let $\one$ be the unit type with point $\pt$.
Then we can define maps $\delta_0, \delta_1 : \one \rightarrow$ sending $\pt$ to 0 and 1 respectively.
Also, we define $\delta : \one + \one \rightarrow I^1$ by $\delta_0 + \delta_1$.

Next we define the homotopy pushout.
\lstset{language=Coq}
\begin{lstlisting}
Inductive $\hpushout$ $(A, B, C : \Type)$ $(f : A \rightarrow B)$ $(g : A \rightarrow C)$ :=
| $\inl$ : $B \rightarrow$ $\hpushout$ $A$ $B$ $C$ $f$ $g$
| $\inr$ : $C \rightarrow$ $\hpushout$ $A$ $B$ $C$ $f$ $g$
| $\glue$ : $\prod (a : A), \inl (f \> a) = \inr (g \> a)$
\end{lstlisting}
Note the similarities with the construction of the pushout.
Also, we can define the homotopy colimit as a higher inductive type in much the same way.
\lstset{language=Coq}
\begin{lstlisting}
Inductive $\hocolim$ $(F : \mathbb{N} \rightarrow \Type)$ $(f : \prod (n : \mathbb{N}, F \> n \rightarrow F(n+1)))$ :=
| $\inc$ : $\prod (n : \mathbb{N}), F \> n \rightarrow$ $\hocolim$ $F$ $f$
| $\com$ : $\prod (n : \mathbb{N}) (x : F \> n), \inc \> n \> x = \inc \> (n+1) \> (f \> n \> x)$
\end{lstlisting}

\section{The Approximator}
\label{sec:approximator}
Let us assume that some higher inductive type $H$ is given.
In order to construct $H$ as a colimit, we first need to give the approximations in the colimit, and for that we define the \emph{approximator}.

Before giving the definition, let us think about how it should be given.
By Ad\'amek's theorem, every inductive type can be given as a colimit.
An inductive type $T$ is given by a polynomial functor $F$ and a constructor $c : F \> T \rightarrow T$, and then $T$ is the colimit of the sequence
\[
\xymatrix
{
	\zero \ar[r] & F \> \zero  \ar[r] & F(F \> \zero) \ar[r] & \ldots
}
\]
To understand what this does, let us assume that $F \> X = 1 + X$, so that $T = \mathbb{N}$.
This means we have two inclusions  $\xymatrix{1 \ar[r] & 1+X}$ and  $\xymatrix{X \ar[r] & 1+X}$, and we call them $0_C$ and $S_C$ respectively.
At every step we formally add for each $x : X$ a successor $S_C \> x$, and we add $0_C$.
Repeatedly applying this construction to the empty type $\zero$ gives the natural numbers $\mathbb{N}$.

For higher inductive types one would like to do a similar construction.
The first difference is that instead of starting with nothing and adding each constructor at every step, we start with the nonrecursive constructors, and add recursive constructors at every step.
Also, since extra equalities might be present in the higher inductive type, we need to make identifications during the construction.
Rather than just adding points at every step, we also need to glue the right paths.

To understand what should be done more precisely, let us consider an example.
\lstset{language=Coq}
\begin{lstlisting}
Inductive $\mathbb{N}_2$ :=
| $0$ : $\mathbb{N}_2$
| $S$ : $\mathbb{N}_2 \rightarrow \mathbb{N}_2$
| $p$ : $S(S \> 0) = 0$
\end{lstlisting}
The first approximation just has a constructor $0$, and after that we add a constructor for $S \> 0$ to obtain the second approximation.
In the third approximation, which we call $F \> 3$, we found inhabitants $0$, $S \> 0$, and $S(S \> 0)$, and now we can make the first identification.
To do so, we take the following homotopy pushout
\[
\xymatrix
{
	\one + \one \ar[d]_{0 + S(S \> 0)} \ar[r]^-{\delta} & I^1 \ar[d]\\
	F \> 3\ar[r] & P \> 3
}
\]
to obtain the actual third approximation $P \> 3$, and we continue our construction with that one.
Note that to glue during the $n$th step, we need refer to elements from the $(n-2)$th step.
So, in order to do the identification of the $n$th step, one needs to refer back to a previous approximation.

Note that one always has to go back a fixed number of steps due to the usage of constructor terms.
By extending the syntax, one can also think of examples where one needs to go back an arbitrary amount of steps.
For that, we consider the following example.
\lstset{language=Coq}
\begin{lstlisting}
Inductive $\mathbb{N}_2$ :=
| $0$ : $\mathbb{N}_2$
| $S$ : $\mathbb{N}_2 \rightarrow \mathbb{N}_2$
| $p$ : $\prod (n : \mathbb{N}), S^n \> 0 = 0$
\end{lstlisting}
This example might seem like it is allowed in the syntax, but it is not.
The term $S^n$ is defined as a polymorphic function which needs the type as argument, and that is not allowed with constructor terms.
In this extension the construction will be more complicated, and the construction is not predicative, so this will not be considered in this paper.

Let us make this idea formal, and for that we start with a higher inductive type which is given as follows.
\lstset{language=Coq}
\begin{lstlisting}[label={lst:HIT}]
Inductive $H$ :=
| $c_{\nonrec}$ : $A_{\nonrec} \rightarrow H$
| $c_{\rec}$ : $A_{\rec} \> H \rightarrow H$
| $p_i$ : $\prod (x : B_i \> H), t_i = r_i$ $\quad (i = 1, \ldots, m)$
\end{lstlisting}
Note that the nonrecursive and recursive point constructors are separated in this definition.
The first approximation will be given using the nonrecursive constructors.
\lstset{language=Coq}
\begin{lstlisting}
Inductive $H_{\nonrec}$ :=
| $c_{\nonrec}'$ : $A_{\nonrec} \rightarrow H_{\nonrec}$
\end{lstlisting}

Next we need to generate the other approximations, and that will be done in two steps.
First, we note that types can be extended with a recursive constructor.
\lstset{language=Coq}
\begin{lstlisting}
Inductive $H_{\rec}$ $(P : \Type)$:=
| $c_{\rec}'$ : $A_{\rec} \> P \rightarrow H_{\rec} \> P$
\end{lstlisting}
To do the identifications, we need to be able to interpret the constructor terms.
For that we use that each constructor terms only uses each constructor  a finite amount of times, and thus there is a maximum number $n \geq 1$ of times a constructor is used.
In order to define the approximator, we start with a type $H_{\Con}^n$ in which all the constructor terms can be interpreted.
\lstset{language=Coq}
\begin{lstlisting}
Inductive $H_{\Con}^n$ $(P : \Type)$:=
| $\term$ : $P + H_{\rec} \> P + \ldots + H_{\rec}^n \> P \rightarrow H_{\Con} \> P$
\end{lstlisting}

If $n$ is clear from the context, then we shall not write it down.

\begin{lemma}
Suppose, we have a constructor term $t$ such that $x : F \> T \Vdash t : G \>T$ which uses at most $n$ constructors, and that we have a map $c_{\nonrec}' : A_{\nonrec} \rightarrow P$.
Then $t$ induces a map $\overline{t} : F \> P \rightarrow G \> H_{\Con}^n$ by replacing the constructors $c_{\nonrec}$ and $c_{\rec}$ by $c_{\nonrec}'$ and $c_{\rec}'$ respectively.
\end{lemma}

\begin{proof}
We use induction on the form of the constructor term.
\begin{itemize}
	\item $t = a$ with $a : B$ and $B$ does not use $T$. Then we define $\overline{t} \> y = a$.
	\item $t = x$ with $x : F \> T$. Then we define $\overline{t} \> y = \inn_P y$.
	\item $t = c_{\nonrec} a$ with $a : A_{\nonrec}$. Then we define $\overline{t} \> y = c_{\nonrec}' a$.
	\item $t = c_{\rec} r$ with $r : A_{\rec} \> T$ where $r$ uses at most $n-1$ constructors. 
			  By induction we have a map $\overline{r} : F \> P \rightarrow A_{\rec} \> H_{\Con}^{n-1}$.
			  Then we define $\overline{t} \> y = c_{\rec}' (\overline{r} \> x)$.
	\item For the rules for the projection, pairing and injection it is trivial. \qedhere
\end{itemize}
\end{proof}

Now we have enough to define the approximator.
During the construction one also needs to pay attention to coherency.
For example, in $P$ we have $c_{\nonrec}$, so in $H_{\Approx} \> P$ we might have duplicates of some terms.
By tweaking the pushout a little bit, one can easily solve these coherency problems.

\begin{definition}
Let a higher inductive type $H$ be given as before, and let the type $H_{\Con}$ be defined as before.
Then we define the \emph{approximator} $H_{\Approx}$, which has a parameters $P, Q$ and $f : Q \rightarrow P$, of $H$ the following pushout
\[
\xymatrix
{
	((\one + \one) \times B_i \> P) + H_{\Con} \> Q \ar[d]_{\overline{t_i} + \overline{r_i} + H_{\Con} \> f} \ar[rr]^-{\delta + \inl} & & (I^1 \times B_i \> P) + H_{\Approx} Q  \ar[d]^{\inr}\\
	H_{\Con} \> P \ar[rr]_{\inl} & & H_{\Approx} \> P
}
\]
Note that this can be written as the homotopy pushout.
\end{definition}

\begin{definition}
In the setting as described, we simultaneously define a sequence of approximations $F : \mathbb{N} \rightarrow \Type$ to $H$ and maps $f : \prod(n : \mathbb{N}), F \> n \rightarrow F(n+1)$ as follows
\begin{align*}
F \> 0 &= H_{\nonrec},\\
F \> 1 &= H_{\Approx} \> (F \> 0) \> \zero \> (\zero\frec (F \> 0))\\
F(n+2) &= H_{\Approx} \> (F(n+1)) \> (F \> n) \> (f \> n) \qedhere
\end{align*}
For the maps maps $f \> n : F \> n \rightarrow F(n + 1)$, note that we always have the following sequence of maps
\[
\xymatrix@1
{
	P \ar[r] & H_{\Con} \> P \ar[r] & H_{\Approx} P.
}
\]
Taking $P$ to be $F \> n$, then we have $F (n+1) = H_{\Approx} (F \> n)$, and thus the composition gives the map $F \> n \rightarrow F(n + 1)$.
\end{definition}

\section{The Rules}
\label{sec:rules}
Now we have defined an object $\hocolim \> F \> f$, which is supposed to interpret the higher inductive type.
In order to finish the proof of Theorem \ref{thm:three-hit}, we need to show that it satisfies the rules.
This means that we have to make functions which interpret the introduction rules, and an eliminator such that the computation rules are satisfied.
We will do this step by step, and refer to lemmata in Section \ref{sec:Lemmata} when needed.

\subsection{Introduction Rules}
In order to show that this is the desired type, we first show that it has the correct introduction rules.
These come in three flavors: the nonrecursive and the recursive points, and the paths.

Let us start by defining a map $A_{\nonrec} \rightarrow \hocolim \> F \> f$ which gives the introduction rule for the nonrecursive point constructor.
Since $F \> 0$ is defined by $H_{\nonrec}$, which has a constructor $c'_{\nonrec} : A_{\nonrec} \rightarrow H_{\nonrec}$, this can be defined by the following composition.
\[
\xymatrix
{
	A_{\nonrec} \ar[rr]^-{c'_{\nonrec}}
		& & F \> 0 \ar[rr]^-{\inc \> 0}
		& & \hocolim \> F \> f
}
\]

Next we show that we also have the recursive point constructor meaning that we have a map $A_{\rec} \> (\hocolim \> F \> f) \rightarrow \hocolim \> F \> f$.
This is slightly more complicated, and for that we first need a lemma which says that colimits over $\mathbb{N}$ commute with polynomials.

\begin{lemma}
\label{lem:colim_commute_poly}
The types $A \> (\hocolim \> F \> f)$ and $\hocolim \> (A \circ F) \> (A \> f)$ are isomorphic for all polynomials $A$.
\end{lemma}

\begin{proof}

\end{proof}

Now we will construct the map $A_{\rec} \> (\hocolim \> F \> f) \rightarrow \hocolim \> F \> f$, and by Lemma \ref{lem:colim_commute_poly} it suffices to make a map $\hocolim \> (A_{\rec} \circ F) \> (A \> f) \rightarrow \hocolim \> F \> f$.
For this we use the recursion rule of $\hocolim$, and we start with the following string of maps
\[
\xymatrix
{
		A_{\rec}(F \> n) \ar[r]^-{c_{\rec}'} 
		& H_{\rec}(F \> n) \ar[rr]^{\term \circ \iota_2} &
		& H_{\Con}(F \> n) \ar[r]^-{\inl}
		& H_{\Approx}(F \> n) = F(n + 1)
}
\]
where the map $H_{\rec}(F \> n) \rightarrow H_{\Con}(F \> n)$ is the inclusion and the map $H_{\Con}(F \> n) \rightarrow H_{\Approx}(F \> n)$ is the pushout map.
Composing this map with $\inc$, gives maps $A_{\rec}(F \> n) \rightarrow \hocolim \> F \> f$ for all $n : \mathbb{N}$.

Next we need to show the commutativity of the following triangle.
\[
\xymatrix
{
	A_{\rec}(F \> n) \ar[rr]^{A_{\rec}(f \> n)} \ar[dr] & & A_{\rec}(F(n+1)) \ar[dl] \\
	& \hocolim \> F \> f &
}
\]
Before we continue, let us recall that inductive types are functors.
Suppose, we have an inductive type $T$ with a parameter $P$, then a function $f : P \rightarrow Q$ gives a function $T \> f : T \> P \rightarrow T \> Q$.
Let us start with the following rectangle
\[
\xymatrix
{
	A_{\rec}(F \> n) \ar[r]^-{c_{\rec}'} \ar[d]_{A_{\rec}(f \> n)} 
		& H_{\rec}(F \> n) \ar[rr]^{\term \circ \iota_2} \ar[d]_{H_{\rec}(f \> n)} &
		& H_{\Con}(F \> n) \ar[d]_{H_{\Con}(f \> n)}\\
	A_{\rec}(F(n+1)) \ar[r]_-{c_{\rec}'} 
		& H_{\rec}(F(n+1)) \ar[rr]_{\term \circ \iota_2} &
		& H_{\Con}(F(n+1))
}
\]
The left square commutes, because by definition of $H_{\rec} \> f$ we have
\[
H_{\rec} \>(f \> n) \> (c_{\rec}' x) = c_{\rec}' (A_{\rec} \> (f \> n) \> x).
\]
For a similar reason the right triangle commutes as well.
Hence, it suffices to show that the following diagram commutes.
\[
\xymatrix
{
	H_{\Con}(F \> n) \ar[d]_{H_{\Con} \> (f \> n)} \ar[r]^-{\inl}
		& H_{\Approx}(F \> n) \ar[d]^{f(n+1)}\\
	H_{\Con}(F(n+1)) \ar[r]_-{\inl}
		& H_{\Approx}(F(n+1))
}
\]
Note that $H_{\Approx}(F(n+1)) = F(n+2)$ and $H_{\Approx}(F \> n) = F(n+1)$.
Hence, this diagram commutes, because of the pushout.

Next we need to define the introduction rules for the paths.
Let us start by making a map $\sum_{i = 1}^m I^1 \times B_i \> (\hocolim \> F \> f) \rightarrow \hocolim \> F \> f$.
Since homotopy colimits commute with polynomials, it suffices to make a map $\hocolim \> (I^1 \times B_i \> F) \> (I^1 \> B_i \> f)  \rightarrow \hocolim \> F \> f$.

For this we use the eliminator of the homotopy colimit.
First, we need to make for all $n$ a map $I^1 \times B_i \> (F \> n)  \rightarrow \hocolim \> F \> f$.
By definition we have the following string of maps
\[
\xymatrix
{
	I^1 \times B_i \> (F \> n) \ar[rr]^-{\inr} & & F(n+1) \ar[rr]^-{\inc(n+1)} & & \hocolim \> F \> f.
}
\]
Next we need to show that the following diagram commutes.
\[
\xymatrix
{
	I^1 \times B_i \> (F \> n) \ar[rr]^-{\inr} \ar[d]_{I^1 \times B_i \> (f \> n)} & & F(n+1) \ar[d]^{f(n+1)}\\
	I^1 \times B_i \> (F(n+1)) \ar[rr]^-{\inr} & & F(n+2)
}
\]
This can be proven by induction on $B_i$, and using the definition of $I^1 \times B_i \> (f \> n)$.
We can use the path $\refl$.
All in all, we get a map $p_i : \hocolim \> (I^1 \times B_i \> F) \> (I^1 \times B_i \> f) \rightarrow \hocolim \> F \> f$

Now we have the following diagram
\[
\xymatrix
{
	2 \times B_i(\hocolim \> F \> f) \ar[r] \ar[dr]_{\overline{t_i} + \overline{r_i}} & I^1 \times B_i(\hocolim \> F \> f) \ar[r] & \hocolim \> (I^1 \times B_i \> F) \> (I^1 \times B_i \> f) \ar[dl]\\
	& \hocolim \> F \> f
}
\]

$2 \times B_i(\hocolim \> F \> f)$ is isomorphic to $\hocolim \> (2 \times B_i \> F) \> 2 (\times B_i \> f)$.

Induction to second, so we have $(0, x)$ (similar for $1$) and $x : F \> n$.
This gets mapped to $(0, x)$ which is equal to $t_i$ via the path $\glue (\inl (0, x))$.

Remaining:
\begin{equation*}
\begin{split}
\com_*(\ap \> (\inc \> n) \> \glue (\inl (0, x))) = \ap \> (\inc (n+1))) \>\glue (\inl (0, f \> n \> x))\\
(\ap \> \overline{t_i} \> \com)^{-1} \conc \ap \> (\inc \> n) \> \glue (\inl (0, x)) \conc \ap \> p_i \> \com
\end{split}
\end{equation*}
$(\ap \> \overline{t_i} \> \com)^{-1}$ is like $\ap \> (\inc \> n) \> (\ap \> c \> \ldots (\ldots \ap \> \text{coherence path}))$, because of the computation rule of $c_i$.

This is the coherence path which should be added on level $n$.


\subsection{Elimination Rule}
For the next step we define the right eliminator for $\hocolim \> F \> f$. 
For this we suppose that we have
\[
Y : \hocolim \> F \> f \rightarrow \Type
\]
\[
c_{Y, \nonrec} : \prod (a : A_{\nonrec}), Y(c_{\nonrec})
\]
\[
c_{Y, \rec} : \prod (x : A_{\rec} \> \hocolim \> F \> f), \overline{A_{\rec}} \> Y \> x \rightarrow Y(c_{\rec} \> x)
\]
\[
q_{Y, i} : \prod (x : B_i \> H)(h_x : \overline{B_i} \> Y \> x), \hatt_i =_{p_j \, x} \hatr_j
\]
In order to make a map $\prod (x : \hocolim \> F \> f), Y \> x$, we use the induction principle of $\hocolim \> F \> f$, and for that we need to make maps $\prod (x : F \> n), Y(\inc \> n \> x)$.

First we need to make a map $\prod (x : F \> 0), Y(\inc \> 0 \> x)$.
Recall that $F \> 0$ was defined to be $H_{\nonrec}$ which only has a constructor $c'_{\nonrec} : A_{\nonrec} \rightarrow H_{\nonrec}$.
So, let us assume that we have $a : A_{\nonrec}$.
Since $c_{\nonrec} \> a = \inc \> 0 \> (c\_{\nonrec} a)$, it suffices to find an inhabitant of $Y(c_{\nonrec} \> a)$, and for that we take $c_{Y, \nonrec} \> a$.
Hence, we get a map $\prod (x : F \> 0), Y(\inc \> 0 \> x)$.

Now suppose that we have a map $\prod (x : F \> n), Y(\inc \> n \> x)$, and our goal is to make a map $\prod  (x : F(n+1)), Y(\inc \> (n+1) x)$.
In order to do so, we will first look at how to extend it to a map $\prod (x : H_{\Con} (F \> n)), Y(\inc \> (n+1) \> (\inl \> x))$.

Let us do this in the general case.
Suppose that we have $g : P \rightarrow \hocolim \> F \> f$, and that we have $h : \prod (x : P), Y(g \> x)$.
Our goal is to extended the map into a $h' : \prod (x : H_{\rec} P), Y(c_{Y, \rec} \> x)$, and for that we use $H_{\rec}$ induction.
Note that for each $c'_{\rec} \> x$ with $x : A_{\rec} \> P$ we have the type $Y(c_{\rec} \> (A_{\rec} \> g \> x))$, and thus we have a type family on $H_{\rec}$.
Now let $x : A_{\rec} \> P$ and $y : \overline{A_{\rec}} \> Y \> x$ be given.
Then we need to give an element of the type $Y(c_{\rec} \> (A_{\rec} \> g \> x))$ for which we take
\[
c_{Y,\rec} \> (A_{\rec} \> g \> x) \> y
\]
Now we can also extend the map to $H_{\Con}^n \> P$, because we can define this map on each component.
On the component $P$ it is just $h$, and on the other components we define it via extension.

Next we need to extend the map to $H_{\Approx} \> P \> Q$.
For that we need to make a map $h' : \prod (x : I^1 \times B_i \> P), Y()$

Let $(t, x) : I^1 \times B_i \> P$, and now we use $I^1$ induction.
If $t = 0$, we map it to $\hatt_i (x)$, and for $t = 1$ we map it to $\hatr_i (x)$.
Now the path $q_{Y, i}$ gives the image of $\seg$, and that finishes the induction.

For the commutativity of the diagram note that by definition the dependent map from $H_{\Con} \> P$ to $Y$ sends the constructor term $t_i$ to the lift $\hatt_i$.
This is because the map replaces each $c_{\rec}$ by $c_{Y, \rec}$ which is precisely how the lift was constructed.


Next we do it for the coherency.

Induction on the level (so, ultimately some kind of simultaneous induction?)

Then induction on the depth.

Base case: just induction assumption.

Step case: then it can be worked out from the diagram.



To finish the proof, we need to give the image for the path $\com$.
More concretely, we need to show that $\inc \> n \> x$ and $\inc \> (n+1) \> (f \> x)$ are mapped to the same element.
The map $f : F \> n \rightarrow F(n + 1)$ is defined by the pushout as follows
\[
\xymatrix
{
	((\one + \one) \times B_i \> (F(n+1))) + H_{\Con} \>  (F \> n) \ar[d]_{\overline{t_i} + \overline{r_i} + H_{\Con} \> f} \ar[rr]^-{\delta + \inl} & & (I^1 \times B_i \> (F(n+1))) + F(n+1)  \ar[d]^{\inr}\\
	H_{\Con} \> (F(n+1)) \ar[rr]_{\inl} & & F(n+2)
}
\]
Since the map to $Y$ was defined as the pushout map, it will commute automatically.

All in all, we have acquired a map $\prod (x : \hocolim \> F \> f), Y \> x$, and this way we defined the right eliminator for $\hocolim \> F \> f$.
We shall call the eliminator $H\frec$.

\subsection{Computation Rules}
Lastly, we show that this eliminator also satisfies the computation rules.
First, we prove that for each $t : A_{\rec}(\hocolim \> F \> f)$ that $H\frec(c_{\rec} \> t) \equiv f_i \> t \> (\overline{A_{\rec}} H\frec t)$.
Again we use that colimits commute with polynomials.

Let $n : \mathbb{N}$ and $x : A_{\rec}(F \> n)$.
Then by the computation rules we have
\begin{equation*}
\begin{split}
H\frec(c_{\rec} \> x) 
&\equiv H\frec(\inc \> (n+1) \> (c_{\rec}' x)) \\
&\equiv c_{Y,\rec} \> (A_{\rec} \> (\inc \> n) \> x) \> (\overline{A} \> H\frec \> x)\\
\end{split}
\end{equation*}
Hence, we can always take $\refl$ to be the path.
This will also give an image for $\com$, and thus the computation rules for the points are satisfied.

Note that the this computation rule is a propositional equality.
This is logical, because it is proven all $x : A_{\rec}(F \> n)$.
However, for closed terms, we have a definitional equality.
This is because $c_{\nonrec} \> a$ for $a : A_{\nonrec}$ is defined to be$\inc \> 0 \> (c_{\nonrec}' \> a)$.
All the closed terms are thus inhabitants of some $F \> n$, and since at every step the equalities are definitional, we can conclude that for closed terms the equality is definitional.

Now we show the computation rules for the paths.


\section{Lemmata}
\label{sec:Lemmata}

\begin{lemma}
$\hocolim \> (\lambda n . A) \> \operatorname{Id}$ is $A$.
\end{lemma}

\begin{lemma}
Colimits commute with coproducts.
\end{lemma}

\begin{lemma}
Colimits commute with products.
\end{lemma}

\section{Conclusion and Further Work}



\nocite{*}

%%
%% Bibliography
%%

%% Either use bibtex (recommended), 

\bibliography{lipics-v2016-sample-article}

%% .. or use the thebibliography environment explicitely



\end{document}
